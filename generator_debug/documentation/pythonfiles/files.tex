\documentclass[12pt]{article}
\usepackage[margin=1in]{geometry}
\usepackage{hyperref}
\usepackage{listings}
\usepackage{xcolor}

\title{Composites Filament Winder Documentation}
\author{Original Code: Rylan Andrews, Dalton Reichenbach} 
\date{Troubleshooting: Evan Mullins}

\begin{document}

\maketitle
\newpage

\section{Composites Filament Winder Project Overview}

The project contains code and documentation for the filament winder.  

\subsection{Project Structure}
The project is organized into two main folders:

\begin{enumerate}
    \item \textbf{generator}: Program used on a desktop/laptop to create a configuration file that is transferred to the Arduino. This is where wind angle, plies, and other settings are configured.
    \item \textbf{documentation}: Contains documentation for the winder (currently empty).
\end{enumerate}

\subsection{Python Files in Generator}
There are seven Python files with varying functionality:

\begin{itemize}
    \item \texttt{definitions.py} --- Internal definitions for the rest of the code.
    \item \texttt{gui.py} --- GUI version of \texttt{main.py} using Tkinter; primary interface for G-code generation.
    \item \texttt{helper.py} --- CLI helpers: printing help, loading schedules, generating G-code, and calculations.
    \item \texttt{load.py} --- Loads wind schedule data from a JSON file.
    \item \texttt{main.py} --- Original CLI program (does not set default feed rate).
    \item \texttt{planner.py} --- Computes internal math to plan the filament winding paths.
    \item \texttt{winder.py} --- Simulates the winder and generates G-code commands.
\end{itemize}

Only two files are directly used to output G-code:

\begin{itemize}
    \item \texttt{gui.py} --- Updated GUI version.
    \item \texttt{main.py} --- Original CLI version (does not automatically add feed rate line).
\end{itemize}

\subsection{Key Helper Files}

\paragraph{definitions.py} Defines wind properties, including:

\begin{itemize}
    \item \textbf{Wind angle} --- Angle between mandrel axis and tow direction.
    \item \textbf{Number of starts} --- Number of start positions per pattern.
    \item \textbf{Lock angle} --- Angle through which mandrel turns at the end of each pass.
\end{itemize}

\paragraph{helper.py} Provides CLI functions:

\begin{itemize}
    \item \texttt{help} --- Print command menu.
    \item \texttt{load} --- Load wind schedule.
    \item \texttt{generate} --- Generate G-code.
    \item \texttt{calculator} --- Compute valid wind parameters.
\end{itemize}

\paragraph{load.py} Loads JSON configuration files into the program.

\paragraph{planner.py} Contains all mathematical calculations for winding paths.

\paragraph{winder.py} Simulates the physical winder and outputs G-code.  

\textbf{Axis definitions:}

\begin{itemize}
    \item X-axis: carriage motion along mandrel (inches)
    \item Z-axis: mandrel rotation (degrees)
    \item Y-axis: currently unused
\end{itemize}

\textbf{Key G-code outputs:}

\begin{itemize}
    \item \texttt{G20} --- Set units to inches
    \item \texttt{G01 F100.0} --- Feed rate (missing in \texttt{main.py})
    \item \texttt{G92 X0 Z0} --- Set axis position
    \item \texttt{G28} --- Move home (currently not called)
\end{itemize}

\subsection{Issue: Feed Rate Line Missing in CLI}
The F-line (\texttt{G01 F…}) is not generated in \texttt{main.py} because:

\begin{itemize}
    \item \texttt{Winder} constructor sets \texttt{currentFeedRate} to the default.
    \item \texttt{setFeedRate} only appends a line if feed rate changes or \texttt{force=True}.
    \item CLI did not call \texttt{setFeedRate} with \texttt{force=True}.
\end{itemize}

\textbf{Solution:}

\begin{enumerate}
    \item Update \texttt{Winder.setFeedRate}:
\begin{verbatim}
def setFeedRate(self, f, force=False):
    if force or self.currentFeedRate is None or f != self.currentFeedRate:
        self.gcode.append(f"G01 F{round(f,3)}")
        self.currentFeedRate = f
\end{verbatim}
    \item Remove feed rate from constructor or set \texttt{currentFeedRate=None}.
    \item Always call \texttt{setFeedRate(..., force=True)} before generating G-code:
\begin{verbatim}
machine = winder.Winder()
machine.setFeedRate(defaultFeedRate, force=True)
gcode = planner.planWind(schedule, machine)
\end{verbatim}
    \item Optionally call \texttt{machine.moveHome()} to add homing line (\texttt{G28}).
\end{enumerate}

\end{document}